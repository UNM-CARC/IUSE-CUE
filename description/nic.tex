\section{NIC Activities}
\label{sec:nic}
%ssc I changed the title of this section, but I'm not sure how to Latex it so that it updates in the full draft. Sorry!
The NMIX+CS project will use a NIC model for developing and coordinating activities across and between the project partners. As described above, project partners have a long history of successful collaboration as well as current collaborations through projects like NM CS4All, S-STEM, and the NM SMART Grid Center. In addition, a number of NIC-supporting activities are planned around four areas: Shared Goal, Theory of Improvement, Common Metrics, and Network Organization and Management.

\paragraph{Shared Goal:} Project partners have the shared goal of developing an integrated state-wide framework for CS+X education programs with an emphasis on broadening participation in computing. NIC elements to support that goal include:
\begin{tightitemize}
        \item Demonstrated success in collaboration by project partners on broadening participating in computing projects;
        \item A near-peer mentoring program described in Section~\ref{sec:framework:support}; and,
        \item Cross-institutional course sharing, specifically aligning the content of and offering cross-enrollment in CS undergraduate courses (\textit{see the letter of support from the provosts of UNM, NMSU, and NMT}). %and senior-level big data computer science classes, 
        %building on commitments from the provosts at each institution to investigate mechanisms for cross-institutional course sharing. 
        A cross-institutional mechanism for sharing graduate courses has been already implemented and used among the three institutions and will be considered for this project.
\end{tightitemize}

\paragraph{Theory of Improvement:} Our initial theory of improvement for the NMIX+CS project is described in the logic model, see Table~\ref{tab:logic-model}. It draws on diverse experts from both computer science and non-computer science disciplines across three institutions with a variety of innovations ranging in scale from new course modules to new courses. The robust evaluation and assessment plan will provide feedback that will help to inform and refine our theory of improvement.

\paragraph{Common Metrics:} The NIC will employ shared metrics through a coordinated assessment and evaluation plan developed by a dedicated internal evaluator and implemented across the entire NIC. The metrics the NIC seeks to optimize are:
\begin{tightitemize}
\item The number of students applying for and admitted to NMIX+CS programs
\item NMIX+CS student graduation and job placement rates 
\item The number and percentage of students from under-represented groups enrolled in and 
graduating from NMIX+CS programs, particularly women, Native American, and Hispanic students.
\end{tightitemize}
The associated assessment and evaluation plans we will use is described in Section~\ref{sec:assessment}.

\paragraph{NIC Organization and Management:} Network management is especially important in the initiation phase~\cite{nic1}. The interaction between the members of the NIC will be based on frequent communication through a number of different structures, including face-to-face meetings, virtual meetings, email, and the sharing of evaluation and EAB reports. Individual subteams working on specific course modules and/or courses will meet frequently. Face-to-face meetings for the entire NIC team will occur with a frequency of at least one time per semester, at centrally-located meeting facilities available to the NIC members (e.g., the UNM Sevilleta Field Station, the New Mexico Consortium, NM EPSCoR). The management team, composed of the PIs from each institutions will meet monthly. The NIC will also employ meeting technologies like Zoom and Slack to facilitate communication and collaboration.

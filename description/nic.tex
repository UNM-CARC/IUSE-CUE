\section{NIC Activities}
%ssc I changed the title of this section, but I'm not sure how to Latex it so that it updates in the full draft. Sorry!
The NMIX+CS project will use a NIC model for developing and coordinating activities across and between the project partners. As described above, project partners have a long history of successful collaboration as well as current collaborations through projects like NM CS4All, S-STEM, and the NM SMART Grid Center. In addition, a number of NIC-supporting activities are planned around four areas: Shared Goal, Theory of Improvement, Common Metrics, and Network Organization and Management.

\textbf{Shared Goal:} Project partners have the shared goal of developing an integrated state-wide framework for CS+X education programs with an emphasis on broadening participation in computing. Elements to support that goal include:
\begin{itemize}
        \item Demonstrated success in collaboration by project partners on broadening participating in computing projects
        \item A near-peer mentoring program 
        \item Commitments from provosts at each institution to share curricula and investigate formalizing collaboration through mechanisms like cross-institution course sharing
\end{itemize}
\textbf{Theory of Improvement:} Our initial theory of improvement for the NMIX+CS project is described in the logic model, see Figure XX. It draws on diverse experts from both computer science and non-computer science disciplines across three institutions with a variety of innovations ranging in scale from new course modules to new courses. The robust evaluation and assessment plan will provide feedback that will help to inform and refine our theory of improvement.

\textbf{Common Metrics:} The NIC will employ shared metrics through a coordinated assessment and evaluation plan developed by a dedicated internal evaluator and implemented across the entire NIC.

\textbf{NIC Organization and Management:} Network management is especially important in the initiation phase (Ref:Russell, J. L., Bryk, A. S., Dolle, J. R., Gomez, L. M., Lemahieu, P. G., \& Grunow, A. (2017). A Framework for the Initiation of Networked Improvement Communities. Teachers College Record, 119(5), n5.). The interaction between the members of the NIC will be based on frequent communication through a number of different structures, including face-to-face meetings, virtual meetings, email, and the sharing of evaluation and EAB reports. Individual subteams working on specific course modules and/or courses will meet frequently. Face-to-face meetings for the entire NIC team will occur with a frequency of at least one time per semester, at centrally-located meeting facilities available to the NIC members (e.g., the UNM Sevilleta Field Station, the New Mexico Consortium, NM EPSCoR). The management team, composed of the PIs from each institutions will meet monthly. The NIC will also employ meeting technologies like Zoom and Slack to facilitate communication and collaboration.

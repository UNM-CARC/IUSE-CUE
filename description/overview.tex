\section{Overview}
\label{sec:overview}

% PGB - Add a BPC section to the outline
% Move some pieces of text around
% Make assignements for everyone
We propose a Networked Improvement Community (NIC) of the three New Mexico research universities to research
systematic integration of education in computational thinking into the undergraduate STEM programs at our
institutions. In particular, we seek to create the New Mexico Inter-disciplinary X + CS (NMIX+CS) program,
a set of mutually supporting inter-disciplinary undergraduate programs (e.g. an inter-disciplinary minor or
certificate program) for undergraduate students seeking to add a computational component to their undergraduate
degree. Effectively support computing education for and by diverse populations
and disciplines is a key goal of this effort; this is vital because the three institutions have diverse
characters and student populations as state's general research, land grant, and technical universities, all of which
are also Minority Serving Institutions (MSIs) with significant under-represented (e.g. Hispanic
and Native American) populations. As such, our key metrics we seek to improve through this NIC are:
\begin{itemize}
\item Number of students applying for and admitted to NMIX+CS programs
\item NMIX+CS graduation and job placement rates
\item Number and percentage of students from under-represented students enrolled in and graduating from
  NMIX+CS programs, particularly women, Native American, and Hispanic students.
\end{itemize}

We plan to develop and prototype parallel NMIX+CS programs at UNM, NMSU, and NMT in a diverse set of shared disciplines,
starting with:
Computer Science across the three institutions; Criminology and Criminal Justice and UNM and NMSU; Psychology at NMSU and NMT; Biomedical Engineering at UNM; Nursing at NMSU; and Management at NMT.
This effort will include sharing courses and course materials including remote instruction and
cross-institution course sharing, comparison of assessment results across institutions and classes, and
sharing of best practices and  experiences between the institutions where appropriate.

These programs will leverage multiple educational approaches to integrating computing education into all levels of
our undergraduate STEM curricula, including but not limited to:
\begin{itemize}
\item Course modules for introductory STEM discipline classes that \emph{expose} students to computing concepts
  in the context the STEM discipline being studied;
\item Introductory STEM computing courses that provide students the tools to \emph{use} computing in their discipline,
  building on their prior exposure to computing in STEM concepts
\item Computing strands on computational techniques, problem solving, and computing ethics for integration
  throughout STEM classes to build \emph{mastery} of computing topics and computational thinking by STEM
  students;
\item STEM + CS for use in STEM research methods classes to \emph{capstone} computing education
  curricula for STEM students.
\end{itemize}
Through these mechanisms, each student in an NMIX+CS program will be receive computing education
across at least 5 different courses, sufficient for, for example, an undergraduate certificate or inter-disciplinary
minor, depending on the appropriate mechanism at each institution.

The NMIX+CS programs will be advised by an external board comprised of representatives of local industry and government partners
invested in computing education at our institutions; we have already tentatively identified several such members. This board will
be provided program evaluation results to help guide its advise on program direction.  Prof.~Amir Hedayati-Mehdiabadi at
the UNM Organization, Information and Learning Sciences (OILS) interdisciplinary program will lead program evaluation
of the NMIX+CS programs in collaboration with personnel at the other institutions.

\paragraph{Intellectual Merit:} Our proposed research encompasses four key contributions:
\begin{enumerate}
\item First, we will...
\item Second, we will...
\item Third, we will...
\item Fourth, we will...
\end{enumerate}

\paragraph{Broader Impact:} The proposed research will have significant impact
both in the broader high performance computing community as well as directly
in educational activities. First, the techniques we propose will...

In addition, we will also integrate the developed techniques into the UNM
Computer Science and Library Sciences curricula and recruit to increase the 
participation of women and under-represented minorities in research. This work includes...

Finally, will also explicitly work to broaden the participation of women and under-represented
minorities in computer and library sciences as part of this project... This effort  will
leverage the New Mexico Universities' positions Hispanic serving research institutions and integrate and expand multiple diversity programs across all three institutions.


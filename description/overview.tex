\section{Overview}
\label{sec:overview}

% PGB - Add a BPC section to the outline
% Move some pieces of text around
% Make assignements for everyone
The three UNM research universities propose to create the New Mexico Inter-disciplinary X + CS (NMIX+CS) program, a Networked Improvement Community (NIC) focused on the systematic integration of computational thinking into the undergraduate STEM programs at our institutions. In particular, we seek to create,
a set of mutually supporting inter-disciplinary undergraduate programs (e.g. an inter-disciplinary minor or
certificate program) for undergraduate students seeking to add a computational component to their undergraduate
degree. The program will include a combination of educational approaches, mentoring and recruitment activities, and inter-institutional collaborations all designed to determine the best way to support computing education for New Mexico students.

Effectively support computing education for and by diverse populations
and disciplines is a key goal of our proposed effort; this is vital because the three institutions in the community have diverse
characters and student populations. In particular, the proposed consortium includes the fstate's general research institution (UNM), land grant institution (NMSU), and technical research institution (NMT), all of which
are also Minority Serving Institutions (MSIs) with significant under-represented (e.g. Hispanic
and Native American) populations. We have identified a diverse set of disciplines at the three institutions in which we will pilot integrating computational techniques ranging from Nursing to Psychology to Biomedical Engineering. This will allow us to to create and assess the range for  educational approaches needed to best address the diverse educational needs and student populations in these programs.

Our overall educational approach to integrating computing education into different disciplines is based on a flexible curriculum that provides multiple mechanisms to instruct students in using computing techniques in the context of their discipline. This includes:
\begin{tightitemize}
\item Course modules for introductory STEM discipline classes that \emph{expose} students to computing concepts
  in the context the STEM discipline being studied;
\item Introductory STEM computing courses that provide students the tools to further \emph{expose} students to STEM computing techniques and teach them to \emph{use} these techniques to solve problems in multiple STEM disciplines;
\item Computing strands on computational techniques, problem solving, and computing ethics  to build \emph{mastery} of computing topics and computational thinking that can be either be integrated into a sequence of STEM research methods classes or form a stand-alone ``Computational Methods in X" class, depending on discipline needs; and
\item Discipline-specific STEM + CS projects that can integrated into either advanced project-oriented general computer science classes or into discipline-specitif befor use in STEM research methods classes to \emph{capstone} computing education
  curricula for STEM students.
\end{tightitemize}
The results of these educational approaches will be measured through an internal evaluation process and an external advisory board will assess these results of these evaluations to improve the program.

These mechanisms and evaluations will be augmented by multiple mechanisms to recruit and support a diverse population of STEM students with the overall goal of broaden participation in computing. This includes program-specific faculty and peer mentoring, extensive use existing student support and STEM student services, and integration with each institution's programs for recruiting and supporting minority student populations. In addition, the NMIX+CS NIC includes sharing of courses,
course materials, assessment results, and best practices and experiences across instutitions through multiple mechanisms to increase the collective impact of the proposed educational and broadening activities.

The key metrics NMIX+CS seeks to improve through these educational, broadening participation, and networking/collaboration mechanisms are:
\begin{tightitemize}
\item Number of students applying for and admitted to NMIX+CS programs
\item NMIX+CS graduation and job placement rates
\item Number and percentage of students from under-represented students enrolled in and 
  graduating from NMIX+CS programs, particularly women, Native American, and Hispanic students.
\end{tightitemize}

\subsection{Intellectual Merit} 
NMIX+CS encompasses four key contributions of intellectual merit:
\begin{enumerate}
\item First, we will...
\item Second, we will...
\item Third, we will...
\item Fourth, we will...
\end{enumerate}

\subsection{Broader Impact} 
The proposed educational program will also have significant broader impacts, including:

\section{Overview}
\label{sec:overview}

% PGB - Add a BPC section to the outline
% Move some pieces of text around
% Make assignements for everyone
The three New Mexico (NM) research universities propose to create the \emph{New Mexico Inter-disciplinary X + CS (NMIX+CS)} program, a Networked Improvement Community (NIC) focused on the systematic integration of computational thinking into the undergraduate STEM programs at our institutions. In particular, we seek to create
a set of mutually supporting inter-disciplinary undergraduate programs (e.g., an inter-disciplinary minor, a supplemental major, or
certificate program) for undergraduate students seeking to add a computational component to their undergraduate non-computing
degree. The program will include a combination of educational approaches, mentoring and recruitment activities, and inter-institutional collaborations all designed to determine the best way to support computing education for New Mexico students.

Effectively supporting computing education for and by diverse populations
and disciplines is a key goal of our proposed effort; this is vital because the three institutions in the community have diverse
characteristics and student populations. In particular, the proposed consortium includes the state's Carnegie 1 research institution (the University of New Mexico---UNM), land grant institution (New Mexico State University---NMSU), and technical research institution (New Mexico Institute of Mining and Technology---NMT), all of which
are also Minority Serving Institutions (MSIs) with significant under-represented (particularly Hispanic
and Native American) populations. We have identified a diverse set of disciplines at the three institutions in which we will pilot integrating computational techniques, ranging from Nursing to Psychology to Biomedical Engineering. This will allow us to to create and assess a range of educational approaches to best address the diverse educational needs and student populations in these programs.

Our overall educational approach to integrating computing education into different disciplines is based on a flexible curriculum that provides multiple mechanisms to instruct students in learning computational thinking and computing problem solving techniques in the context of their discipline. This includes:
\begin{tightitemize}
\item Course modules for introductory \emph{non-Computer Science (non-CS)} discipline classes that \emph{expose} students to computing concepts
  in the context of the host discipline;
\item Introductory CS courses for non-CS majors that provide students the tools to further \emph{engage} students in learning  computing techniques and teach them to \emph{use} these techniques to solve problems in multiple non-CS disciplines;
\item Computing strands on computational techniques, problem solving, and computing ethics  to build \emph{mastery} of computing topics and computational thinking that can be either be integrated into a sequence of non-CS research methods classes or form a stand-alone ``Computational Methods in X'' class, depending on discipline needs; and
\item Discipline-specific ``X + CS'' projects that can integrated into either advanced project-oriented general computer science classes or into discipline-specific  research methods classes to \emph{capstone} computing education
  curricula for non-CS students.
\end{tightitemize}
The results of these educational approaches will be measured through an internal evaluation process, while an external advisory board will assess the results of these evaluations to improve the program.

These instruments and evaluations will be augmented by multiple mechanisms to recruit and support a diverse population of non-CS students with the overall goal of broadening participation in computing. This includes program-specific faculty and peer mentoring, extensive use of existing student support and STEM student services, and integration with each institution's programs for recruiting and supporting students from traditionally underrepresented groups. In addition, the NMIX+CS NIC includes sharing of courses,
course materials, assessment results, and best practices and experiences across New Mexico's institutions, through multiple mechanisms to increase the collective impact of the proposed educational and broadening participation activities.

The key metrics NMIX+CS seeks to improve through these educational, broadening participation, and networking/collaboration mechanisms are:
\begin{tightitemize}
\item Number of students applying for and admitted to NMIX+CS programs
\item NMIX+CS graduation and job placement rates
\item Number and percentage of students from under-represented grops enrolled in and 
  graduating from NMIX+CS programs, particularly women, Native American, and Hispanic students.
\end{tightitemize} 

\subsection{Intellectual Merit} 
\paragraph{Patrick to write here}
NMIX+CS encompasses four key contributions of intellectual merit:
\begin{enumerate}
\item Overall framework for X+CS 
\item Pilot integrations of these into a diverse set of STEM disciplines
\item Discipline-specific X+CS ethics modules 
\item Assessment and evaluation of these of these elements
\end{enumerate}

\subsection{Broader Impacts} 
The proposed educational program will also have significant broader impacts, including:
\patrick{Selena to write here. Diversity and educational impact}
\melanie{Numbers of courses developed/revised and student educated is important here}

An important aspect of this project is the development of educational programs that provide computational training to students that might not otherwise consider CS as a viable or interesting option. This will contribute to the expansion of the computing-capable workforce, bringing to the workforce a more diverse talent. In turn, the program will develop a workforce that is better prepared to tackle modern inter-disciplinary challenges, using cutting-edge computing technologies.

The emphasis placed, in our demonstration efforts, on non-CS disciplines  will also contribute to broadening participation  of students from groups that are traditionally underrepresented in computing. For example, Nursing is a traditionally women-dominated discipline \cite{nursing}, while Biomedical Engineering is among the top 5 majors among Hispanic students \cite{biomed}. Thus, targeting our framework to this type of non-CS disciplines will help in further diversifying the computing workforce.

The project will emphasize the adoption of inclusive pedagogy and mentoring practices focused on promoting inclusion and diversity, in order to not only engage students with diverse backgrounds, but also to ensure their progression and success in the proposed curriculum.




\section{Overview}
\label{sec:overview}

The three New Mexico (NM) research universities propose the \emph{New Mexico Inter-disciplinary X + CS (NMIX+CS)} program, a Networked Improvement Community (NIC) focused on the integration of computer science into undergraduate programs at their institutions. This NIC will create a set of mutually supporting inter-disciplinary programs for students seeking to add a computational component to their undergraduate non-computing degree. The project will include a combination of educational approaches, mentoring and recruitment activities, and inter-institutional collaborations designed to develop the best way to support computing education for New Mexico students.

Effectively supporting computing education for and by diverse populations is a key goal of our proposed effort; this is vital because the three institutions in the community have diverse characteristics and student populations.  The proposed consortium includes the state's Carnegie 1 research institution (the University of New Mexico---UNM), land grant institution (New Mexico State University---NMSU), and technical research institution (New Mexico Institute of Mining and Technology---NMT), all of which
are also Minority Serving Institutions (MSIs) with significant under-represented (particularly Hispanic
and Native American) populations. The project includes a diverse set of disciplines at the three institutions in which we will pilot integrating computational techniques, ranging from Nursing to Psychology to Biomedical Engineering. This will enable the assessment of a range of educational approaches to best address the diverse educational needs and student populations in these programs.

Our overall educational approach %to integrating computing education into different disciplines 
is based on a flexible curriculum that provides multiple mechanisms to instruct students in learning computer science skills
%tational thinking and computing problem solving techniques 
in the context of their discipline. This includes:
\begin{tightitemize}
\item Course modules for introductory non-CS discipline classes that \emph{expose} students to computing concepts in the context of the host discipline;
\item Introductory CS courses for non-CS majors that \emph{engage} students in learning computing techniques and teach them to \emph{employ} these techniques to solve problems in multiple non-CS disciplines;
\item Computing strands on computational techniques, problem solving, and computing ethics  to build \emph{mastery} of computing topics and computational thinking, either in a sequence of disciplinary research methods classes or a stand-alone ``Computational Methods in X'' class; and
\item Discipline-specific ``X + CS'' projects that can be integrated into either senior-level CS classes that target CS-literate discipline practitioners or into discipline-specific capstone project classes.
\end{tightitemize}
The results of these educational approaches will be measured through an internal evaluation process, and an external advisory board will assess the results of these evaluations to improve the program.

The key metrics NMIX+CS seeks to improve  are the number of students applying for and admitted to NMIX+CS programs, NMIX+CS graduation and job placement rates, 
and the number and percentage of students from under-represented groups enrolled in and 
graduating from NMIX+CS programs, particularly women, Native American, and Hispanic students. Because of this, the program will include multiple mechanisms to recruit and support a diverse population of non-CS students with the overall goal of broadening participation in computing. This includes program-specific faculty and peer mentoring, extensive use of existing student support and STEM student services, and integration with each institution's programs for recruiting and supporting students from traditionally underrepresented groups. In addition, the NMIX+CS NIC includes sharing of courses,
course materials, assessment results, and best practices and experiences across New Mexico's institutions through multiple mechanisms to increase the collective impact of the proposed educational and broadening participation activities.

\subsection{Intellectual Merit} 
NMIX+CS encompasses four key contributions of intellectual merit:
\begin{tightenumerate}
\item An overall framework for integrating Computer Science Education into other STEM and non-STEM discipline curricula;
\item Pilot integration of this framework into a diverse set of disciplines; 
\item Discipline-specific computing ethics modules for integration into discipline courses; and
\item Systematic assessment and evaluation of these elements.
\end{tightenumerate}

\subsection{Broader Impacts} 
The proposed educational program will also have significant broader impacts, including:
\begin{tightenumerate}
    \item New educational programs that broaden participation in computing by providing training to students that might not otherwise consider CS as a viable or interesting option;
    \item A more diverse workforce that is better prepared to tackle modern inter-disciplinary challenges using cutting-edge computing technologies;
    \item Creation of a state-wide community of practice focused on X+CS education and an infrastructure to enable the sharing of curricula and classes among the three participating institutions. 
\end{tightenumerate}

 




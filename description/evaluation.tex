\section{Program Evaluation and Advisement}

To guide the development of an evaluation plan for the proposed research, we have developed the logic model show in Table~\ref{tab:logic-model}. The model further specifies the specific details of resources, activities, outputs, outcomes, and impact of the project.

\begin{table}[hp!]
\begin{tabular}{|p{1in}|p{1in}|p{1in}|p{1in}|p{1in}|}
    \hline
    \textbf{Resources}& \textbf{Activities}
    
   \textbf{(Outputs)} & \textbf{Short-term Outcomes} 
    & \textbf{Long-term Outcomes} &\textbf{Impact} \\
    \hline
    \textit{Here we can include the list of the departments that will be involved in each institution as well as other resources available }

    &
    Developing new courses (\# of developed courses, \# of enrollment in the courses)

    \vspace{0.25in}
    Identifying and modifying existing relevant courses (\# of identified and modified courses)

    \vspace{0.25in}
    Developing components on ethics to be integrated in current and new courses (\# of developed assignments, activities, course modules)

    & 
    \textbf{For non-CS students:}

    Developing computing knowledge and skills 

    Increased self-efficacy and understanding of the subject matter

    \vspace{0.25in}
    \textbf{For CS students:}
    
    Increased understanding of the broader impacts of CS

    \vspace{0.25in}
    \textbf{For participating institutions:}
    
    \vspace{0.25in}
    \textbf{For participating non-CS departments:}
    
    & 
    Increased student interest in pursuing degrees/ careers which require computational skills and knowledge  

    \vspace{0.25in}
    Increased cooperation between CS and non-CS departments/faculty

    & 
    A diverse NM workforce with computational knowledge and skills \\
    
    % This row is about NIC activities
    Let’s put a row at the bottom to describe the elements related to the NIC—this will help address the solicitation requirements.
    &
    &
    Activities of NIC (meetings, workshops, advisory group meetings)
    &
    Something about NIC across NM?
    &
    \\
\hline
\end{tabular}
\caption{Evaluation Plan Logic Model}
\label{tab:logic-model}
\end{table}

The purpose of the proposed evaluation plan is to determine to what extent and in what ways the proposed program achieves its intended outcomes which include …. The results will be used to improve the program to increase attainment of outcomes. The following questions will guide the evaluation efforts: 
\begin{enumerate}
\item To what extent and in what ways does the program succeed in improving computational thinking among the students? 
\item To what extent and in what ways does the program succeed in increasing the capacity of faculty and programs to address computational thinking in non-CS disciplines? 
\item To what extent and in what ways does the program influence the career development of participants?
\item What are the program impacts on development and diversity of the next generation of individuals with computational thinking skills?
\item What aspects of the program are succeeding? What areas need improvement? 
\end{enumerate}

Mixed methods is the mode of inquiry that is selected for conducting this evaluation. The goal is to understand the program and to determine the attainment of its short-term and long-term outcomes as well as making judgments about its impacts by using a combination of methods and approaches that are appropriate and relevant to the scope and the overall goals of the program.    
Considering the timeline of the program, a formative evaluation plan has been considered to be conducted in each semester in order to inform the program to modify as the program activities are being implemented for the following semesters. The results of the formative evaluation plan at the end of the 18 month period, however, can be used to determine the overall success of the program (summative). 

The details of the data collection plan will be discussed below:
\begin{description}

\item[Data collection method 1:] The data from the course assignments: in partnership with the instructors, a sub set of course assignments will be selected to be analyzed in order to assess the students’ accomplishments as part of the course work. 
\item[Date collection method 2:] The program statistics such as enrollment data, participants’ grades, etc. 
\item[Data collection method 3:] The data from conducting a survey to understand student interests and goals for pursuing the certificate/ enrolling in the course: This questionnaire will include close-ended and open-ended questions to understand students’ perceptions of the program and the possible benefits and also their goals in pursuing the certificate (at the beginning of each semester). 
\item[Data collection method 4:] End of semester questionnaire: This questionnaire will include both close and open-ended questions to gather students’ feedback regarding the course and their progress in the course. 
\item[Data collection method 5:] Interviews with the informants from the participating departments to understand their overall perception of the program and its strengths and weaknesses as perceived by the participating departments (at the end of the 18 month period) 
\item[Date collection method 6:] A focus group of students who completed the program (at the end of the 18 months period) 
\item[Data collection method 7:[] Individual interviews with participants might be considered as needed for clarification of inconsistencies in data. 
\end{description}
Data analysis will include open coding for qualitative data (focus group and open-ended questions) as well as descriptive and simple inferential statistics for quantitative data.  

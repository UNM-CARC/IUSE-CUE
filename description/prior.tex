\section{Prior NSF Support}
\label{sec:prior}

\paragraph{Patrick Bridges:}
\textbf{NSF Grant OAC-1807563, ``CDS\&E: Optimization of Advanced Cyberinfrastructure through Data-driven Computational Modeling''; September 2018--August 2011, \$523,644.}

\noindent\textit{Intellectual Merit:} The goals of this new project are to investigate a new inter-collective-interval based model for quantifying the performance of complete, realistic application/system software configurations. This approach will provide a general approach for understanding a wide range of application/system software performance tradeoffs for both HPC and cloud computing applications and systems using a combination of novel data-centric and statistical performance modeling approaches complementary to the approaches described in this proposal.

\noindent\textit{Broader Impact:} This project seeks to enable the design and deployment of efficient and effective next-generation cyberinfrastructure systems, for example NSF XSEDE systems and NSF Cloud systems. In addition, the project is focusing on integrating computational modeling and measurement into the system software portion of the computer science curriculum, providing students with experience both modeling and measuring the performance of high-end computing systems. 

\noindent The project has just begun, and so does not yet have results to report.

\paragraph{Lisa Broidy:} None to report.

\paragraph{Huiping Cao:} 
\textbf{NSF Grant OIA-1757207, ``New Mexico SMART Grid Center'';September 2018 - August 2023, \$20,000,000 }

\noindent\textit{Intellectual Merit:} Huiping Cao is senior personnel on this  interdisciplinary research center award that is researching a Sustainable, Modular, Adaptive, Resilient, and Transactive (SMART) next-generation electric grid. Specially, the center's goal is  develop an integrated research and education program that provides consumers the ability to decide how to generate, store and manage energy on the existing distribution infrastructure.

\noindent\textit{Broader Impacts:} A key impact of this award is harnessing New Mexico's strengths, institutions, and infrastructure in the energy domain to address the complex goal of modernizing the electricity grid, creating one the largest interdisciplinary groups in the country working on smart grid research while engaging the public through programs about the SMART Grid research. 

\paragraph{Amir Heydayati:} None to report.

\paragraph{Subhasish Mazumdar:}

\paragraph{Melanie Moses:}
\textbf{NSF Grant 1240992, Supplement PI Moses, original PI Lee, ``CS 10K: New Mexico Computer Science for All (NM CSforAll)'', November 2012 - October 2016} 

\noindent \textit{Intellectual Merit:} The NM CSforAll program is a teacher PD and AP CS Principles-aligned high school course that teaches computer modeling and simulation, computational thinking, programming and pedagogy. %It addresses the lack of CS-trained teachers and formal CS courses in NM high schools. Online teacher PD, the high school course, and a freshman non-majors course have been taught in a hybrid flipped-classroom with videos, example code, programming assignments, tests, grading rubrics and extensive curricular material hosted online. Teachers guide their students in programming labs during class time. 
%The curriculum aligns with the CS Principles Big Ideas and Learning Objectives (particularly focused on Impact, Abstraction, Algorithms, Programming, and Data) and the CSTA K-12 Standards \textcolor{red}{REF Lee et al}.
The project demonstrated that computational modeling is an effective way to teach CS content and skills while fostering positive attitudes and student engagement.
%Additionally, a hybrid flipped-classroom course that offers college dual credit can be scaled up to teach CS to a rural, socio-economically disadvantaged and underrepresented population.  
Teachers and students showed significant increases in learning objectives; 100\% of teachers rated the PD very good or excellent. Increasing creative and open-ended design modules is particularly effective for Native and Hispanic students who were found to be significantly more likely than their peers to find CS important for scientific problem solving and a necessary skill for everyone to learn \textcolor{red}{REF Svihla}. Students also made significant improvements on math objectives \textcolor{red}{REF math conf paper}. This work was chosen as a SIGCSE 2017 exemplary paper (top 7.5\% of submissions). \textcolor{red}{REF Lee et al.}

\noindent \textit{Broader Impact:}
51 high school teachers from 18 NM school districts completed NM CSforAll PD and taught over 1300 students; 75\% are underrepresented including 36\% women and 20\% Native American. An undergraduate section of the course was established at UNM as the first computer science course in New Mexico to count toward a core university graduation requirement. By switching recruitment from pre-computer science majors to advertising the course as satisfying the natural science core, enrollment rose from 14 to 42\% women. 
%Through a partnership of CS educators, NM CSforAll contributed to a 2017 New Mexico law that allows CS to count toward NM high school math or science graduation requirements. 
The Swarmathon Project Module, initially developedy by NM CSforAll REUs, was used by over 450 high school students in 23 high schools across the US to compete in the 2017 virtual NetLogo NASA Swarmathon Challenge. 
%NM CSforAll course content is freely distributed (http://cs4all.cs.unm.edu/).

\paragraph{Dimiter Petsev:}
The NSF supported research efforts of Dr. Petsev are in the area of electrokinetic phenomena, nanofluidics, thin liquid film transport, and colloid stability, charged interfaces. The NSF Grants include \textbf{NSF/NIRT Fundamental Understanding of Nanofluidics for Advanced Bioseparation and Analysis, Grant (CTS 0404124, 2004-2008)}; \textbf{UNM/Harvard PREM: Leadership in Biomaterials” (DMR 0611616, 2006-2011)}; \textbf{NSF-Separations and Purification New Generation of Lab on Chip Separators based on Independent Fluid and Analyte Control (CBET 0828900, 2008-2012)}; and \textbf{NSF-CAREER Transport Control in Fluidic Micro and Nanochannels (CBET 0844645, 2009-2014)}.

\noindent\textit{Intellectual Merit:} All NSF funded project (as well as those supported by other sources) involved undergraduate students. Over the last ten years, 15 students work on various research projects. Five students arrived through the NSF/REU program. The intellectual contributions of the students were significant and five were included as co-authors on publications. [1-6] Two more are currently on a submitted paper. This work exposed the students to various aspects of chemical and materials engineering and computational methods. [7]

\noindent\textit{Broader Impact:} Out of all students on these projects, nine were female (three of them were Hispanic and one was Native American). The REU students, hosted by Dr. Petsev, were four females and a male of Hispanic descent.    

\patrick{I will convert these to bibtex references later.}
1. N.J. Carroll, P.F. Crowder, W. Patterson, D.R. Ratnaweera, D. Perahia, P. Atanassov, and D.N. Petsev, Microfluidic synthesis of monodisperse nanoporous oxide particles and control of hierarchical pore structure. ACS Appl. Mater. Interfaces, 2013. 5: p. 3514-3529.

2.	N.J. Carroll, S. Pylypenko, A. Ortiz, B.T. Yonemoto, C. Lopez, P. Atanassov, D.A. Weitz, and D.N. Petsev, Droplet based microfluidics for synthesis of mesoporous silica microspheres MRS Symposium Proceedings, 2010. 1272: p. KK08-02.

3.	N.J. Carroll, S.B. Rathod, E. Derbins, S. Mendez, D.A. Weitz, and D.N. Petsev, Droplet-based microfluidics for emulsion and solvent evaporation synthesis of monodisperse mesoporous silica microspheres. Langmuir, 2008. 24(3): p. 658-661.

4.	S.T. Chang, E. Beaumont, D.N. Petsev, and O.D. Velev, Remotely powered distributed microfluidic pumps and mixers based on miniature diodes. Lab on a Chip, 2008. 8: p. 117-124.

5.	R. Vangara, D.C.R. Brown, D.J. Prakash, D.N. Petsev, and F.v. Swol. Corrosion and surface charge in electric double layers by classical density functional theory. in Proceeding of the 2017 DOD-Allied Nations Technical Corrosion Conference. 2017. Birmongham, Al.

6.	R. Vangara, D.C.R. Brown, F.v. Swol, and D.N. Petsev, Electrolyte solution structure and its effect on the properties of electric double layers with surface charge regulation. J. Colloid Interface Sci., 2017. 488: p. 180-189.

7.	R. Vangara, M. York, K. Stoltzfus, F.v. Swol, and D.N. Petsev, Electric double layers containing multivalent ions. J. Phys. Chem. C, 2018: p. submitted.

\paragraph{Enrico Pontelli:}
\textbf{NSF Grant HRD-1834620, ``NSF INCLUDES Alliance: Computing Alliance of Hispanic-Serving Institutions,'' August 2018-July 2023}

\noindent \textit{Intellectual Merit:}
Co-PI Pontelli is a Senior Personnel of the CAHSI INCLUDES Alliance, where he serves as the Lead of the Southwest Region of the Alliance. CAHSI’s bold, shared vision (referred to as the 20-30 vision) is:  By 2030, Hispanics will represent 20\% or more of those who earn credentials in computing. 
%Credentials are defined as degrees and certifications that lead to gainful employment and advancement in the field. 
CAHSI’s mission is to grow and sustain a networked community committed to recruiting, retaining, and accelerating the progress of Hispanics in computing.
%The goals of the Alliance can be summarized as follows: {\bf (1)}  Alliance Expansion:  Support scale-up of CAHSI INCLUDES Alliance; {\bf (2)}
% Reinforcing Activities on Hispanic Student Success:  Improve efforts to recruit, retain, and advance Hispanics who will be competitive in the computing workforce and academia; {\bf (3)} Data Management and Analytics: Establish shared data management practices to measure 
%              progress of collective efforts;
%              {\bf (4)}
%Knowledge Creation and Dissemination: Create and disseminate knowledge on Hispanic 
%              student success and expertise in computer science.  


\noindent \textit{Broader Impact:}
The INCLUDES Alliance promotes an inclusive educational and research environment for Hispanic students to thrive in pursuit of their computing education and to succeed as a well-trained workforce with advanced degrees in computing disciplines. 
%The unification of forces from a diverse and wide group of Hispanic Serving Institutions, organized in four regions (West---California, Southwest---Arizona, New Mexico, Texas; Southeast---Puerto Rico and Florida; North---Illinois and New York) together with experiences drawn from field-tested initiatives, has the potential to elevate the INCLUDES CAHSI Alliance as a recognized entity that contributes to and influences the national agenda on education. Future expansion to two-year colleges associated with established CAHSI institutions will further expand the collective impact.  
CAHSI INCLUDES will serve as a model for advancing students from other underrepresented groups in higher education. The study on effective organizational structures will provide valuable lessons to all higher education institutions about how to better serve their Hispanic students in computing disciplines specifically and more broadly in STEM. 
\medskip


\paragraph{Son Tran:}
\textbf{Department of Education Grant GAANN P200A180005, ``Training Graduate Students for Research \& Teaching Careers in AI and Cyber-Security,'' October 2018-August 2021}

\noindent \textit{Intellectual Merit:} co-PI Tran is the PI of the grant GAANN P200A180005. The objective of this grant is to encourage more students, particularly from traditionally underrepresented backgrounds, to pursue graduate studies in Computer Science (CS) leading to doctoral degrees, and preparing them for academic careers in education and research. The project will provide specialized research and educational training in those cutting-edge areas of CS that have been identified by the Secretary of education as areas of national need, specifically artificial intelligence (AI) and cybersecurity. 

\noindent \textit{Broader Impact:} The project aims at increasing the number of domestic students pursuing graduate studies in CS. It will have long term impacts on the individuals who participate in this project as well as others who are influenced or in contact with these individuals in the future. It will have immediate impacts on the underrepresented communities (e.g., the Hispanic community) at NMSU, who will have the opportunity to participate in this project.  


\paragraph{Dongwan Shin:}
\paragraph{Mark Samuel:}
\paragraph{Frank Reinow:}
\paragraph{Conni DeBlieck:}
\paragraph{Alexa Doig:}
\paragraph{Dennis Giever:}


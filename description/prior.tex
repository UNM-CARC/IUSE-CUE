\section{Prior NSF Support}
\label{sec:prior}

\paragraph{Patrick Bridges:}
\textbf{NSF Grant OAC-1807563, ``CDS\&E: Optimization of Advanced Cyberinfrastructure through Data-driven Computational Modeling''; September 2018--August 2011, \$523,644.}

\noindent\textit{Intellectual Merit:} The goals of this new project are to investigate a new inter-collective-interval based model for quantifying the performance of complete, realistic application/system software configurations. This approach will provide a general approach for understanding a wide range of application/system software performance tradeoffs for both HPC and cloud computing applications and systems using a combination of novel data-centric and statistical performance modeling approaches complementary to the approaches described in this proposal.

\noindent\textit{Broader Impact:} This project seeks to enable the design and deployment of efficient and effective next-generation cyberinfrastructure systems, for example NSF XSEDE systems and NSF Cloud systems. In addition, the project is focusing on integrating computational modeling and measurement into the system software portion of the computer science curriculum, providing students with experience both modeling and measuring the performance of high-end computing systems. 

\noindent The project has just begun, and so does not yet have results to report.

\paragraph{Enrico Pontelli:}
\textbf{NSF Grant HRD-1834620, ``NSF INCLUDES Alliance: Computing Alliance of Hispanic-Serving Institutions,'' August 2018-July 2023}

\noindent \textit{Intellectual Merit:}
Co-PI Pontelli is a Senior Personnel of the CAHSI INCLUDES Alliance, where he serves as the Lead of the Southwest Region of the Alliance. CAHSI’s bold, shared vision (referred to as the 20-30 vision) is:  By 2030, Hispanics will represent 20\% or more of those who earn credentials in computing. Credentials are defined as degrees and certifications that lead to gainful employment and advancement in the field. CAHSI’s mission is to grow and sustain a networked community committed to recruiting, retaining, and accelerating the progress of Hispanics in computing. The goals of the Alliance can be summarized as follows: {\bf (1)}  Alliance Expansion:  Support scale-up of CAHSI INCLUDES Alliance; {\bf (2)}
 Reinforcing Activities on Hispanic Student Success:  Improve efforts to recruit, retain, and advance Hispanics who will be competitive in the computing workforce and academia; {\bf (3)} Data Management and Analytics: Establish shared data management practices to measure 
              progress of collective efforts;
              {\bf (4)}
Knowledge Creation and Dissemination: Create and disseminate knowledge on Hispanic 
              student success and expertise in computer science.  


\noindent \textit{Broader Impact:}
The proposed INCLUDES Alliance promotes an inclusive educational and research environment for Hispanic students to thrive in pursuit of their computing education and to succeed as a well-trained workforce with advanced degrees in computing disciplines. The unification of forces from a diverse and wide group of Hispanic Serving Institutions, organized in four regions (West---California, Southwest---Arizona, New Mexico, Texas; Southeast---Puerto Rico and Florida; North---Illinois and New York) together with experiences drawn from field-tested initiatives, has the potential to elevate the INCLUDES CAHSI Alliance as a recognized entity that contributes to and influences the national agenda on education. Future expansion to two-year colleges associated with established CAHSI institutions will further expand the collective impact.  CAHSI INCLUDES will serve as a model for advancing students from other underrepresented groups in higher education. The study on effective organizational structures will provide valuable lessons to all higher education institutions about how to better serve their Hispanic students in computing disciplines specifically and more broadly in STEM. 


\medskip
\textcolor{red}{Son:  Not sure if you need this}
\paragraph{Son Tran:}
\textbf{Department of Education Grant GAANN P200A180005, ``Training Graduate Students for Research \& Teaching Careers in AI and Cyber-Security,'' October 2018-August 2021}

\noindent \textit{Intellectual Merit:} co-PI Tran is the PI of the grant GAANN P200A180005. The objective of this grant is to encourage more students, particularly from traditionally underrepresented backgrounds, to pursue graduate studies in Computer Science (CS) leading to doctoral degrees, and preparing them for academic careers in education and research. The project will provide specialized research and educational training in those cutting-edge areas of CS that have been identified by the Secretary of education as areas of national need, specifically artificial intelligence (AI) and cybersecurity. 

\noindent \textit{Broader Impact:} The project aims at increasing the number of domestic students pursuing graduate studies in CS. It will have long term impacts on the individuals who participate in this project as well as others who are influenced or in contact with these individuals in the future. It will have immediate impacts on the underrepresented communities (e.g., the Hispanic community) at NMSU, who will have the opportunity to participate in this project.  

\paragraph{Lisa Broidy:}
\paragraph{Amir Heydayati:}
\paragraph{Melanie Moses:}
\paragraph{Dimiter Petsev:}
\paragraph{Dongwan Shin:}
\paragraph{Subhasish Mazumdar:}
\paragraph{Mark Samuel:}
\paragraph{Frank Reinow:}
\paragraph{Conni DeBlieck:}
\paragraph{Alexa Doig:}
\paragraph{Dennis Giever:}
\paragraph{Huiping Cao:}

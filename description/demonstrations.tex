\section{New Mexico CUE Framework Demonstrations}
\label{sec:pilots}

To research and evaluate the effectiveness of the approach described in the previous sections, we will pilot the use of our proposed educational framework in multiple disciplines at each participating institution. We chose a diverse set of disciplines across the institutions to assess the varying needs of the disciplines. However, we have also chosen two related but not identical disciplines (criminology and criminal justice) to better assesses how more modest differences in disciplinary and student need impact the educational requirements of X + CS programs. 
Table~\ref{tab:pilots} summarizes how each program will integrate the framework described in Section~\ref{sec:approach} into its curriculum. 
\begin{table}[tbh!]
\begin{tabular}{|p{1in}|p{1.75in}|p{1.5in}|p{1.5in}|}
\hline
\hline
     Institution/ & \multicolumn{2}{c|}{Planned Integration in existing courses} & Proposed New Courses\\ 
     \cline{2-3}
     Discipline   & Course Modules & Discipline Projects                   & \\
     \hline\hline
     \textbf{UNM} & & &  \\  
     \hline
     Chemical and Biological Engineering 
                & CBE101 Intro.\ to Chem.\ and Bio.\ Engineering
                
                  CBE253 Chem.\ and Bio. Engineering Computing
                  
                  CBE317 Numerical Methods for Chem.\ and Bio.\ Eng.
                & CS108 Introduction to Computational Science and Modeling
                
                  CBE451 Senior Seminar (Capstone)
                &\\
     \hline
     Criminology & SOC205 Crime, Public Policy, and Crim.\ Justice
     
                  SOC380 Introduction to Research Methods 
                  
                  SOC381 Soc Data Analysis
                  
                  SOC481 Data Analysis
                & CS108 Introduction to Comp.\ Science
                
                  CS485 Introduction to Big Data (Capstone)
                & Note: SOC380, SOC381, and SOC481 will initially be offered as specialized sections for Criminology + CS students\\
     \hline\hline
     \textbf{NMSU} & & & \\
     \hline
     Criminal Justice & %CJS310 Data Analytics for Criminal Justice 1   
                       CJ300 Introduction to Criminal Justice Research
                      & CS458/459 R Programming 
                      & Data Analytics for Criminal Justice 1 \\
                      & CJ301 Advanced Research Methods     %%CJS410 Data Analytics for Criminal Justice 2 %
                      & CS487/488 Machine Learning/Data Mining & 
                        Data Analytics for Criminal Justice 2 \\
     \hline
     Nursing & NURS314 Computer Technology for Nurses    
             & CS171 Introduction to Computer Science & Introduction to CS for Nursing Students\\ 
             & NURS353 Nursing Informatics & CS482/488 Introduction to Data Mining & Health Informatics \\
    \hline\hline
     \textbf{NMT} & & & \\                 
     \hline
     Business Tech.\ \& Management 
                    & SS389 Science and                Technology Policy
                    & CSE107 Introduction to Programming
                    
                    MGT481 Senior Seminar (Capstone)
                    & Computational Approaches for Digital Business\\
     \hline
     Psychology 
                    & PSY151 Human Factors in Science \& Engineering  
                    & CSE107 Introduction to Programming
                    
                    PSY472 Senior Seminar (Capstone)
                    & Cybersecurity with Human in the Loop\\
\hline\hline
\end{tabular}
\caption{Framework Demonstration Integrations by Institution/Discipline}
\label{tab:pilots}
\end{table}

\paragraph{Criminology and Criminal Justice:} As mentioned above, we explicitly chose to examine the integration of computer science education into criminology/sociology at UNM and criminal justice at NMSU to leverage the similarities \emph{and} differences of the two discipline. UNM is integrating the framework primarily as course modules and strands into existing Sociology and CS courses, while NMSU will research using a combination of existing and new classes. All classes will will also include significant ethics components in appropriate course modules. Lectures, guest speakers, projects, and modules in these classes will be focused on helping students think through the real-world application of big data questions in a range of criminological settings (law enforcement, probation/parole/courts/juvenile justice/social welfare) and its potential substantive and ethical utility and challenges.

\paragraph{Nursing:}
%\textcolor{blue}{(from Huiping/Son)}
NMSU Nursing will offer an introductory computer science course based on the existing General Education CS course (CS171)
and an health informatics course for nursing students. 
%The introductory computer course will provide understanding of how computers work, and practical application and programming experience in using computers to solve problems. It will cover broad aspects of the hardware and software of computers. 
Labs in both classes will stress the use of computers in investigating and reporting on data-intensive scientific problems. 
%Experiences in  major software applications includes an introduction to programming, word processing, spreadsheets, databases, presentations, and Internet applications will be emphasized. 
The health informatics course aims at providing nursing students with the basic data analytic and database management skills and for health informatics. Focus will be on the design and implementation of data repositories or healthcare systems (e.g. patient healthcare information) which meet the needs of different constituents of the industry and maintain the privacy and security of the data. Ethical issues and standards on security management in the development of healthcare systems will be considered. 
 
\paragraph{Chemical and Biological Engineering:}
Framework integration in Chemical and Biological Engineering will focus on teaching students modern computational languages and techniques, with a particular emphasis on teaching the fundamentals of Machine Learning (ML) big data analysis. The development of fast and efficient experimental techniques allows for the generation of large amounts of data. Using ML methods allows for identify and extract patterns and relationships that are sometimes hidden in the large amount of information. Specific project examples for use in both existing classes for capstone projects in the CBE Senior Seminar include the identification of phase transitions and outlining phase diagrams\cite{cbe1}, location pollution sources in aquifers~\cite{cbe2}, and understanding the mutation signatures leading to cancers~\cite{cbe3}.


%DA for criminal justice I will cover the introductory materials of the above topics. DA for criminal justice II will cover more advanced materials of the above topics. 

\paragraph{Business Technology \& Management:}
%The Department of Business Technology and Management is a hybrid of business and engineering. Students get a comprehensive experience in market research, conceptual and analytical skills, and technology, partnering with industrial sponsors for real-life management projects. 
%For business majors, data is a crucial resource that, if exploited via appropriate algorithms, can result in competitive advantage through the design and implementation of profitable business plans and sound executive decision making. As technology evolves and myriad data becomes available, canned 'apps' are not enough: data needs to be explored by 'analysts' and the most efficient algorithms discovered or
%devised; deployment of inefficient algorithms can lead to severe losses.  As is well-known, the use of an algorithm named ORION for routing of UPS's trucks in 55,000 routes was estimated to have saved by the end of 2016 10 million gallons of fuel, 100,000 metric tons in CO2 emissions and an estimated \$300 to \$400 million in cost avoidance\footnote{\url{https://www.ups.com/us/en/services/knowledge-center/article.page?name=orion-the-algorithm-proving-that-left-isn-t-right&kid=aa3710c2}}. Already in today's world of digital business, the pricing of consumer products, electricity, airline tickets are all governed by algorithms.

Students in this program will be offered a new \textit{Computational Approaches for Digital Business} after completing an introductory computer science course
%\textit{Introduction to Programming using Python (CSE107)}, 
as well as an introductory business technology and management course with computing course modules embedded (SS389). The course will includes modules that teach data structures, algorithms, and software design topics through, for example, implementing successive approximation algorithms, 
%such as the Newton-Raphson method for finding roots. The next module will consist of 
summarizing data streams, % summarization problems, such as finding the minimum, maximum, histogram, and the frequency of certain data items continuously given an unending stream of data, 
%. Next, the course will cite
solving well-known problems like the inventory storage problem, the allocation problem, and the transportation problem to start discussion, and various other decision support and machine learning problems.
%These  on approaches to algorithmic solutions including the intuitive greedy algorithms, dynamic programming, and linear programming.  Students will use Python to experimentally determine 
%problem sizes that appear to set limits to practical solvability. Next, the course will consider a decision making process which involves individuals pursuing conflicting interests, introducing Game theory and the mini-max principle leading to algorithms for solving simultaneous linear equations and finally converting a game problem  into linear programming thus re-using the programs developed earlier. Finally, it will cover a machine learning problem: either binary classification of data by building a decision tree experimenting with entropy functions or creating an unsupervised cluster of similar patterns by a $k$-means algorithm experimenting with the ramifications of the initial step.

\paragraph{Psychology:}
In this area, NMT aims to integrate computer science education into psychology by offering a new \textit{Cybersecurity with Human in the Loop} for juniors and seniors who have completed an introductory computer science course and an introductory psychology course with computing course modules embedded (PSY151). 
A key goal of the program's modules, strands, and projects will be guided by asking the question of how to make computer security solutions more usable and effective, introducing various attacks such as phishing, spamming, and web spoofing, and examining both technical and psychology-based approaches to solving these challenges.
%, and then discussing various use cases of computer security solutions based on authentication, access control, and privacy control. Later elements will then cover the topic of how to help users better understand the mental processing model in the context of computer security through the use of various psychological and behavioral analysis models such as reasoned action approach and protection motivation theory. 
%Finally it will cover cognitive solutions leveraging data analytics and artificial intelligence.

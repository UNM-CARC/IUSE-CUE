\section{New Mexico CUE Framework Demonstrations}
\label{sec:pilots}

To research and evaluate the effectiveness of the approach described in the previous sections, we will pilot the use of our proposed educational framework in multiple disciplines at each participating institution. We chose a diverse set of disciplines across the institutions to assess the varying needs of the disciplines. However, we have also chosen two related but not identical disciplines (criminology and criminal justice) to better assesses how more modest differences in disciplinary and student need impact the educational requirements of X + CS programs. 
Table~\ref{tab:pilots} summarizes how each program will integrate the framework described in Section~\ref{sec:approach} into its curriculum. 
\begin{table}[tbhp]
\begin{tabular}{|p{1in}|p{1.75in}|p{1.5in}|p{1.5in}|}
\hline
\hline
     Institution/ & \multicolumn{2}{c|}{Planned Integration in existing courses} & Proposed New Courses\\ 
     \cline{2-3}
     Discipline   & Course Modules & Discipline Projects                   & \\
     \hline\hline
     \textbf{UNM} & & &  \\  
     \hline
     Chemical and Biological Engineering 
                & CBE101 Intro.\ to Chem.\ and Bio.\ Engineering
                
                  CBE253 Chem.\ and Bio. Engineering Computing
                  
                  CBE317 Numerical Methods for Chem.\ and Bio.\ Eng.
                & CS108 Introduction to Computational Science and Modeling
                
                  CBE451 Senior Seminar (Capstone)
                &\\
     \hline
     Crimiology & SOC205 Crime, Public Policy, and Crim.\ Justice
     
                  SOC380 Introduction to Research Methods 
                  
                  SOC381 Soc Data Analysis
                  
                  SOC481 Data Analysis
                & CS108 Introduction to Computational Science and Modeling
                
                  CS485 Introduction to Big Data (Capstone)
                & Note: SOC380, SOC381, and SOC481 will initially be offered as specialized sections for Criminology + CS students\\
     \hline\hline
     \textbf{NMSU} & & & \\
     \hline
     Criminal Justice & 
                      & 
                      & Data Analytics for Criminal Justice 1
                      
                        Data Analytics for Criminal Justice 2 \\
     \hline
     Nursing & & & \\
    \hline\hline
     \textbf{NMT} & & & \\                 
     \hline
     Business Tech.\ \& Management & & & \\
     \hline
     Psychology & & & \\
\hline\hline
\end{tabular}
\caption{Framework Demonstartion Integrations by Institution/Discipline}
\label{tab:pilots}
\end{table}

\paragraph{Criminology and Criminal Justice:} As mentioned above, we explicitly chose to examine the integration of computer science education into crimoinology/sociology at UNM and criminal justice at NMSU to leverage the similarities \emph{and} differences of the two discipline. UNM is integrating the framework primarily as course modules and strands into existing Sociology and CS courses, while NMSU will research using a combination of existing and new classes. All classes will will also include significant ethics components in appropriate course modules. Lectures, guest speakers, projects, and modules in these classes will be focused on helping students think through the real-world application of big data questions in a range of criminological settings (law enforcement, probation/parole/courts/juvenile justice/social welfare) and its potential substantive and ethical utility and challenges.

\paragraph{Nursing:}
\textcolor{blue}{(from Huiping/Son)}
At NMSU, we will offer an introductory computer and an health informatics course for nursing students. 

The introductory computer course will provide understanding of how computers work, and practical application and programming experience in using computers to solve problems. It will cover broad aspects of the hardware and software of computers. Labs will be developed stressing the use of computers in investigating and reporting on data-intensive scientific problems. Experiences in  major software applications includes an introduction to programming, word processing, spreadsheets, databases, presentations, and Internet applications will be emphasized. This course will leverage on the existing General Education Computer Science course (CS171) course at NMSU.    

The goal of the health informatics course is to provide nursing students with the basic data analytic and database management skills for health informatics. Focus will be on the design and implementation of data repositories or healthcare systems (e.g. patient healthcare information) which meet the needs of different constituents of the industry (e.g., patients, doctors, insurance companies, etc.) and maintain the privacy and security of the data. Ethical issues and standards on security management in the development of healthcare systems will be considered. 
 
\paragraph{Chemical and Biological Engineering:}
Framework integration in Chemical and Biological Engineering will focus on teaching students modern computational languages and techniques, with a particular emphasis on teaching the fundamentals of Machine Learning (ML) big data analysis. The development of fast and efficient experimental techniques allows for the generation of large amounts of data. Using ML methods allows for identify and extract patterns and relationships that are sometimes hidden in the large amount of information. Specific project examples for use in both existing classes for capstone projects in the CBE Senior Seminar include the identification of phase transitions and outlining phase diagrams\cite{cbe1}, location pollution sources in aquifers~\cite{cbe2}, and understanding the mutation signatures leading to cancers~\cite{cbe3}.


%DA for criminal justice I will cover the introductory materials of the above topics. DA for criminal justice II will cover more advanced materials of the above topics. 

\paragraph{Business Technology \& Management:}
%The Department of Business Technology and Management is a hybrid of business and engineering. Students get a comprehensive experience in market research, conceptual and analytical skills, and technology, partnering with industrial sponsors for real-life management projects. 
For business majors, data is a crucial resource that, if exploited via appropriate algorithms, can result in competitive advantage through the design and implementation of profitable business plans and sound executive decision making. As technology evolves and myriad data becomes available, canned 'apps' are not enough: data needs to be explored by 'analysts' and the most efficient algorithms discovered or
devised; deployment of inefficient algorithms can lead to severe losses.  As is well-known, the use of an algorithm named ORION for routing of UPS's trucks in 55,000 routes was estimated to have saved by the end of 2016 10 million gallons of fuel, 100,000 metric tons in CO2 emissions and an estimated \$300 to \$400 million in cost avoidance\footnote{\url{https://www.ups.com/us/en/services/knowledge-center/article.page?name=orion-the-algorithm-proving-that-left-isn-t-right&kid=aa3710c2}}. Already in today's world of digital business, the pricing of consumer products, electricity, airline tickets are all governed by algorithms.

At NMT, we will offer a new \textit{Computational Approaches for Digital Business} for juniors and seniors in Business and Technology Management. Students can take this course after they complete an introductory computer science course, \textit{Introduction to Programming in Python (CSE107)}, as well as an introductory business technology and management course with computing course modules embedded. The course will start by implementing successive approximation algorithms such as the Newton-Raphson method for finding roots. The next module will consist of data stream summarization problems such as finding the minimum, maximum, histogram, and the frequency of certain data items continuously given an unending stream of data. Next, the course will cite well-known problems like the inventory storage problem, the allocation problem, and the transportation problem to start discussion on approaches to algorithmic solutions including the intuitive greedy algorithms, dynamic programming, and linear programming.  Students will use Python to experimentally determine 
problem sizes that appear to set limits to practical solvability. Next, the course will consider a decision making process which involves individuals pursuing conflicting interests, introducing Game theory and the minimax principle leading to algorithms for solving simultaneous linear equations and finally converting a game problem  into linear programming thus re-using the programs developed earlier. Finally, it will cover a machine learning problem: either binary classification of data by building a decision tree experimenting with entropy functions or creating an unsupervised cluster of similar patterns by a $k$-means algorithm experimenting with the ramifications of the initial step.

\paragraph{Psychology:}
At NMT, we also aims to integrate computer science education into psychology by offering a new \textit{Computational Neuroscience} course for juniors and seniors in Psychology. Students can take this course after they complete an introductory computer science course, \textit{Introduction to Programming in Python (CSE107)}, as well as an introductory psychology course with computing course modules embedded.
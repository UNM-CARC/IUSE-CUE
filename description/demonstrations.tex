\section{New Mexico CUE Framework Demonstrations}
\label{sec:pilots}

To realize the approach described in the previous section, we propose
research on ...

\subsection{Criminology and Criminal Justice}
\label{sec:demo:crim}

As a first general subarea to examine how to best combine computer science education with STEM disciplines, we will examine how to integrate the existing educational elements described in~\ref{sec:approach} into the criminology/sociology curriculum at UNM and the criminal justice curriculum at NMSU. We explicitly chose these two disciplines as a pair of related pilots because their similarities \emph{and} differences will allow us to better assess the strengths and weaknesses of different approaches to integrating the overall NMIX+CS approach into STEM curricula. For example, both disciplines often analyze qualitative and quantitative crime data to inform public policy decision making, but the goals of these analyses (XXX vs. YYY) may differ significantly. In addition, the two departments and institutions have somewhat different curricular structures. 

\paragraph{Criminology:} We will integrate the NMIX+CS curriculum into the UNM criminiology curriculim with a focus on the utility and use of big data for criminology. This integration will begin in  will begin on a pilot basis with students who self-select into the program. These students will take additional coursework in computer science (?) that covers ???. They will also attend a weekly/monthly (?) workshop that focuses explicitly on topics related to big data in criminology including related ethical issues. The workshop will be facilitated by faculty members in both criminology and computer science and will include guest speakers from the local criminal justice and computer science community to help students think through the real-world application of big data questions in a range of criminological settings (law enforcement, probation/parole/courts/juvenile justice/social welfare) and its potential substantive and ethical utility and challenges.

Should the demand for the pilot program prove to be high and it meets key learning objectives and outcomes, we will consider ways to expand the program and integrate it more formally into the criminology curriculum. This might include a one semester course on big data on criminology that covers similar material as the pilot project workshops but is open to more students and might, once vetted, become required for the criminology major. 

\paragraph{Criminal Justice:}
\textcolor{blue}{(from Huiping/Son)  At NMSU, we will offer one module on introductory computing concepts, which can be integrated into an introductory course in health informatics or other disciplines.  
This module will provide understanding of how computers work, and practical application and programming experience in using computers to solve problems. It will cover broad aspects of the hardware and software of computers. Labs will be developed stressing the use of computers in investigating and reporting on data-intensive scientific problems. Experiences in  major software applications includes an introduction to programming, word processing, spreadsheets, databases, presentations, and Internet applications will be emphasized. 
}

\textcolor{blue}{The goal of the introductory course on data analytics for criminal justice students is to provide students with the basic programming 
skills in a programming language suitable for the data analytics (e.g., the R language). Basic programming structures, operating data files, and creating graphs for data visualization will be emphasized in this course.  }

\textcolor{blue}{The advanced data analytics course for criminal justice students will built on the first course and provide the basic skills for fundamental data analysis tasks, including exploring datasets, writing scripts, and conducting basic statistical analysis. The course will be taught using the same language introduced in the introductory course (the R language) and data from the 
area of criminal justice. }

\subsection{Nursing}
\label{sec:research:nursing}
At NMSU, we will offer two courses on data analytics (DA) for criminal justice and one introductory course to health informatics for nursing students. 

\textcolor{blue}{The goal of the health informatics course is to provide nursing students with the basic data analytic and database management skills for health informatics. Focus will be on the design and implementation of data repositories or healthcare systems (e.g. patient healthcare information) which satisfy the needs of different constituents of the industry (e.g., patients, doctors, insurance companies, etc.) and maintain the privacy and security of the data. Ethical issues and standards on security management in the development of healthcare systems will be considered. }

\subsection{Chemical and Biological Engineering}

%DA for criminal justice I will cover the introductory materials of the above topics. DA for criminal justice II will cover more advanced materials of the above topics. 

\subsection{Psychology}
\label{sec:research:psy}

\subsection{Business Technology \& Management}
The Department of Business Technology and Management is a hybrid of business and engineering. Students get a comprehensive experience in market research, conceptual and analytical skills, and technology, partnering with industrial sponsors for real-life management projects. 

\section{Broadening Participation}
\melanie{to lead writing here}
Computer Science is one of the least diverse majors with 15\% women and XXX\% URM nationally. NMIX+CS is poised to dramatically increase the diversity of students with CS training by 1) leveraging our highly diverse student populations and 2) taking computing into other disciplines that have more diverse student populations. For example, Nursing is primarily women and Biomedical Engineering is the most ethnically diverse of Engineering majors at UNM. By offering CS+X courses, we will dramatically expand the number and diversity of students in CS courses. This builds on our successful track record from NM CSforAll (over 1500 students in 5 years, 75\% from underrepresented groups) and CS 0 \ep{can you report numbers and diversity}, and we further integrate CS into disciplines that both need CS skills and have large numbers of students underepresented in traditional CS majors.

\melanie{Can and should we report diversity numbers from our current CS programs? What can we say about numbers and diversity of students in the + X majors??}

\dongwan{to list here}

\ep{some ideas here}
CAHSI provides evidence-based practices that are focused on promoting success of Hispanic students in computing. As part of the mentoring process we could introduce some of these practices in the deployment of new curricular. One idea could be to introduce Peer-Led Team Learning (PLTL) in the new courses and perhaps use the peer leaders across the different institutions (e.g., using collaborative virtual environments) or in the least share the training of peer leaders across the institutions. 

CAHSI can also be used as a dissemination instrument---once we have a good model for NIC and CS+X in HSIs, we could hold workshops within the different CAHSI INCLUDES regions and in turn present the model at the CAHSI Symposium (associated to HENAAC).
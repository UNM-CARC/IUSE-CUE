\section{Educational Activities}
\label{sec:approach}
% Course Development
% BPC - Mentoring/Recruitment - Leverage existing recruiting mechanisms, mentoring participant costs

The overall structure of the educational program we propose for integrating computing education with diverse STEM disciplines is based on four flexible direct education elements shown in Figure~\ref{fig:structure}.  These educational elements are designed to be customized to the needs of the STEM discipline into which they are being integrated and the students of that discipline.  These elements also include integrated ethics instruction components focused on ethics in computing problems relevant to the STEM discipline into which computing instruction is being integrated. Finally, each educational element is augmented with cross-cutting student mentoring and support programs designed to increase student success and retention at all steps of the curriculum. 

The remainder of this section overviews our proposed direction instruction approach, describes how ethics trainings will be integrated with the instructional elements, and discusses the student mentoring and support programs which will be integrated with the proposed instructional elements.  Pilot integrations of these this approach into multiple STEM disciplines are presented in section~\ref{sec:pilots}.

\subsection{Direct Instruction Elements}

\subsubsection{Introductory Course Modules}
\patrick{Course modules for introductory STEM discipline classes that \emph{expose} students to computing concepts in the context the STEM discipline being studied;}
\paragraph{Goals:}
\paragraph{Instuctional Approach:}
\paragraph{Example Integration:}
\ep{will draft this}

\subsubsection{Core Computing Classes}
\patrick{Introductory computing courses that provide  students the tools to further \emph{expose} students to computing techniques and teach them to \emph{use} these techniques to solve problems in multiple disciplines;}
\melanie{to write here}

\subsubsection{Expertise-Building Disciplinary Computing Course Strands}
\paragraph{Goals:} The goal of this educational element is to provide a set of extended instructional modules (strands) to teach students mastery the use of computing techniques in their discipline. These students has completed core computer science classes sufficient to gain programming experience suitable for their disciplinary need, these strands will provide students with extended exposure to the use of computer science technique for solving common disciplinary compute- and/or data-intensive problems. 

\paragraph{Structure:} We will develop a set of four-week strands that cover key computer science topic areas through the lens of problems in a specific discipline. Because of the data-centric nature of most of the disciplines on which we focus, we plan to initially focus on course modules that teach fundamentals of data structures, algorithm analysis, object-oriented programming, parallel programming, data analysis, and computing ethics through discipline-specific examples. In addition, we plan to design the modules so that the four weeks of content can be split into overlapping two or tree week modules that can be used in successive classes in a sequence to provide students multiple exposures to each topic. The exact topics, length, and structure of each strand to facilitate customization will be determined during pilot development (Section~\ref{sec:pilots}) and refined using our proposed assessment and evaluation (Section~\ref{sec:assessment}) framework.

\paragraph{Curricular Integration:} In most cases we expect these strands to be integrated into computing-enhanced disciplinary research methods classes. As mentions above, our default structure will be to split each four-week strand into two overlapping three-week modules. We anticipate that three four-week strands would be split across two research methods classes in the UNM sociology/criminology program, for example starting with the SOC380 Introduction to Research Methods and continuing in SOC481 Data Analysis. This will provide each class 6 weeks of relevant computing education to integrate into the existing curriculum. In other cases, however, we will research combining 4 strands into a single class, for example to create a single Nursing Informatics course as NMSU. 

\subsubsection{Capstone X + CS Projects}
\patrick{Discipline-specific X + CS projects that can integrated into either advanced project-oriented general computer science classes or into discipline-specific  before use in research methods classes to \emph{capstone} computing education  curricula for students.}
\patrick{to write here}

\textcolor{blue}{[From Huiping], AT NMSU, the capstone project will allow the students to work on a criminal justice/nursing problem by utilizing DA and Information System techniques. The students will be advised by two faculty, one from computer science and the other from criminal justice/nursing.
}

\subsection{Ethics Training Integration}
In advancing computing as a science and profession, ethical considerations are a critical component in the educational process (Stahl, Timmermans \& Mittelstadt, 2016). Previous studies have argued the importance of integrating ethics throughout computer science and engineering curricula rather than in standalone courses (Newberry, 2004; Yale Weltz, 1998). Incorporating ethics across the curriculum helps students understand the link between the practice in the field and its positive and negative impacts and also see ethical considerations as part of computer science, rather than an add-on (Newberry, 2004; Pantazidou, 1999; Yale Weltz, 1998). The application of computing technologies in different disciplines, despite their significant benefits, involves the possibilities for wrongdoing, e.g., biased algorithms and record tracking systems that influence negatively the lives of minorities and the poor and treat them unjustly (O’Neil, 2016). This necessitates the development of professionals who not only possess computational skills, but also are aware of ethical and social implications of the systems they will design or use (Connolly, 2011, Stahl et al., 2016).

In a collaborative effort and as an integral part of the programs, we will develop a number of ethics modules that will be incorporated into existing courses within partner programs. In the development of these ethics modules we will follow the Association for Computing Machinery (ACM) Code of Ethics and Professional Conduct (Acm.org, 2018). In addition to developing modules to address computer science issues relevant to specific disciplines covered in the proposed courses, we propose the use of case-based discussions to engage students in the process of ethical decision making and raise their awareness about the ethical implications of computing practice in their future professional lives. Based on the content of each CS course, a real-world scenario will be developed along with a discussion protocol, deployable in an online or face-to-face format, in which students will answer questions and comment on their peers’ reasoning and arguments. These discussions will not only make students aware of the ethical issues within the field, but also will help them see the ethical scenarios from different perspectives and  realize the complexity of the ethical issues and the importance of addressing them properly. In addition, course instructor(s) will gain insights into the student ideas and ethical understanding which can be used in developing interventions to improve instruction for future offerings of the course.

Acm.org. (2018). ACM code of ethics and professional conduct. Retrieved from: https://www.acm.org/code-of-ethics.
Connolly, R. W. (2011, June). Beyond good and evil impacts: rethinking the social issues components in our computing curricula. In Proceedings of the 16th annual joint conference on Innovation and technology in computer science education (pp. 228-232).
Newberry, B. (2004). The dilemma of ethics in engineering education. Science and Engineering Ethics, 10(2), 343-351.
O’Neil, C. (2016). Weapons of Math Destruction: How Big Data Increases Inequality and Threatens Democracy, New York, NY: Crown Publishing Group.
Pantazidou, M., \& Nair, I. (1999). Ethic of care: Guiding principles for engineering teaching \& practice. Journal of Engineering Education, 88(2), 205-212.
Stahl, B. C., Timmermans, J., \& Mittelstadt, B. D. (2016). The ethics of computing: A survey of the computing-oriented literature. ACM Computing Surveys (CSUR), 48(4), 55.
Values in technology and disclosive computer ethics in The cambridge handbook of information and computer ethics 
Yale Weltz, E. (1998). A staged progression for integrating ethics and social impact across the computer science curriculum. Computers \& Society, 28(1).


\patrick{I will fix the references later.} 

\subsection{Student Support Activities}
\patrick{I need someone else to write here. Peer mentoring is the main thing to highlight here, but also mention the individual programs that will be coordinated through the NIC.``Leverage and coordinate existing programs''}
\dongwan{to write here}

\ep{NMSU components}[from Mari] The Computing Alliance of Hispanic-Serving Institutions in NSF’s Inclusion across the Nation of Communities of Learners of Underrepresented Discoverers in Engineering and Science program (CAHSI INCLUDES) focuses on the promoting computer science education at primarily Hispanic-Serving Institutions. As the southwest region’s lead institution, New Mexico State University’s CAHSI will serve an integral role in providing student support services through the Peer-Led Team Learning (PLTL) model. Students will engage in the learning process by participating in small-group problem solving under the direction of a trained peer leader. In order to facilitate a broader participation and meet the educational needs of underrepresented groups, particularly of Hispanics and women in X + CS through computational thinking, CAHSI will utilize members of the CAHSI Club and Young Women in Computing (YWiC) to serve as peer mentors. Additionally, CAHSI scholars and YWiC will actively recruit students and perform outreach with the following academic specific-programs to support students: Sigma Theta Tau-Pi Omega Nursing Honor Society, Student Nursing Association, Nursing Advising Center, Alpha Phi Sigma Criminal Justice Society, and the Criminal Justice Mentoring Center. The NMSU system offers additional assistance that mentor, advise, and tutor students: American Indian Program, Black Programs, Campus Tutoring Services, Central for Academic Advising and Student Support, Chicano Programs, Military and Veterans Programs, Sexual \& Gender Diversity Resource Center, Student Accessibility Services, TRiO Programs, and TRiO STEM-H Student Support Services. 
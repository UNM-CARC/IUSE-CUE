\section{Educational Activities}
\label{sec:approach}
\patrick{I'll write this}
% Course Development
% BPC - Mentoring/Recruitment - Leverage existing recruiting mechanisms, mentoring participant costs

The overall structure of the educational program we propose for integrating computing education with diverse STEM disciplines is based on four flexible direct education elements shown in Figure~\ref{fig:structure}.  These educational elements are designed to be customized to the needs of the STEM discipline into which they are being integrated and the students of that discipline.  These elements also include integrated ethics instruction components focused on ethics in computing problems relevant to the STEM discipline into which computing instruction is being integrated. Finally, each educational element is augmented with cross-cutting student mentoring and support programs designed to increase student success and retention at all steps of the curriculum. 

The remainder of this section overviews our proposed direction instruction approach, describes how ethics trainings will be integrated with the instructional elements, and discusses the student mentoring and support programs which will be integrated with the proposed instructional elements.  Pilot integrations of these this approach into multiple STEM disciplines are presented in section~\ref{sec:pilots}.

\subsection{Direct Instruction Elements}

\subsubsection{Introductory Course Modules}
\patrick{Course modules for introductory STEM discipline classes that \emph{expose} students to computing concepts in the context the STEM discipline being studied;}
\paragraph{Goals:}
\paragraph{Instuctional Approach:}
\paragraph{Example Integration:}
\ep{will draft this}

\subsubsection{Core Computing Classes}
\patrick{Introductory computing courses that provide  students the tools to further \emph{expose} students to computing techniques and teach them to \emph{use} these techniques to solve problems in multiple disciplines;}
\melanie{to write here}

\subsubsection{Expertise-Building Course Strands}
\patrick{Computing strands on computational techniques, problem solving, and computing ethics to build \emph{mastery} of computing topics and computational thinking that can be either be integrated into a sequence of research methods classes, used to create a stand-alone ``Computational Methods in X" class, or as disciplinary examples general computing classes, depending on student,  discipline, and institutional needs;}

\patrick{to write here}
\subsubsection{Capstone X + CS Projects}
\patrick{Discipline-specific X + CS projects that can integrated into either advanced project-oriented general computer science classes or into discipline-specific  before use in research methods classes to \emph{capstone} computing education  curricula for students.}
\patrick{to write here}

\textcolor{blue}{[From Huiping], AT NMSU, the capstone project will allow the students to work on a criminal justice/nursing problem by utilizing DA and Information System techniques. The students will be advised by two faculty, one from computer science and the other from criminal justice/nursing.
}

\subsection{Ethics Training Integration}
\patrick{I need someone else to write here. Dennis and Amir} 

\subsection{Student Support Activities}
\patrick{I need someone else to write here. Peer mentoring is the main thing to highlight here, but also mention the individual programs that will be coordinated through the NIC.``Leverage and coordinate existing programs''}
\dongwan{to write here}

\ep{NMSU components}
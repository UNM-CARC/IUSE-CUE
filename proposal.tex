%
% NSF Proposal
%
\documentclass[11pt]{article}
\usepackage{multicol}

%\usepackage{fullpage}
\usepackage{geometry}
\geometry{
  top=1.0in,            % <-- you want to adjust this
  inner=1.0in,
  outer=1.0in,
  bottom=1.0in,
  headheight=3ex,       % <-- and this
  headsep=2ex,          % <-- and this
}
\usepackage{palatino,mathpazo}
\usepackage{pgfgantt}
\usepackage{times}
\usepackage{xspace}
\usepackage{wrapfig}
\usepackage{caption}
%\usepackage{subfigure}
\usepackage{subcaption}
\usepackage{url}
\usepackage{xcolor}     % For 'revision' marks in draft versions
\usepackage{titlesec}
\usepackage{caption}   % more control over caption settings (e.g. font)
\usepackage{rotating}  % allow height=, etc. in graphics
\usepackage{pdfpages}
\usepackage{ifthen}
\usepackage[plainpages=false]{hyperref}
\usepackage{verbatim}
\usepackage{listings}   % To pretty print code
\usepackage{fancyvrb}
\usepackage{multicol}
\usepackage{mathtools}
\usepackage{fancyhdr}

%\usepackage{babel}
\usepackage{paralist, tabularx}

\usepackage{enumitem} % help with spacing {AS}

\usepackage{graphicx}  % \includegraphics defn
\DeclarePairedDelimiter\ceil{\lceil}{\rceil}
\DeclarePairedDelimiter\floor{\lfloor}{\rfloor}

\lstset{language= C}
\lstset{framexleftmargin= 3mm, rulesepcolor= \color{black}}
%\lstset{numberstyle=\tiny, numbers=left, numbersep= 5pt, xleftmargin= 12pt}
\lstset{firstnumber= 1, stepnumber= 1}
\lstset{showstringspaces= false}
\lstset{gobble=0}
\lstset{linewidth=\textwidth}
\lstset{keepspaces= true}
\lstset{showspaces= false}

\hypersetup{colorlinks=false,
        pageanchor=true,
        pdfpagemode=None,
        plainpages=false,
        pdfview=FitH,
        pdfstartview=FitH}

\urlstyle{same}
%\renewcommand\url{\begingroup\urlstyle{sf}\Url} % put url font in sf
\renewcommand{\floatpagefraction}{0.8}
\parindent=2.0em
\advance\baselineskip by-1pt
\parskip=0.1ex minus 0.1ex

%Tight itemizes and enumerates
\makeatletter
\def\tightitemize{\ifnum \@itemdepth >3 \@toodeep\else \advance\@itemdepth
\@ne
\edef\@itemitem{labelitem\romannumeral\the\@itemdepth}%
\list{\csname\@itemitem\endcsname}{\setlength{\topsep}{-\parskip}\setlength{\leftmargin}{\labelwidth}\setlength{\parsep}{0in}\setlength{\itemsep}{0in}\setlength{\parskip}{0in}\def\makelabel##1{\hss\llap{##1}}}\fi}
\let\endtightitemize =\endlist
\def\tightenumerate{%
\ifnum \@enumdepth >\thr@@\@toodeep\else
\advance\@enumdepth\@ne
\edef\@enumctr{enum\romannumeral\the\@enumdepth}%
\expandafter
\list
\csname label\@enumctr\endcsname
{\setlength{\topsep}{-\parskip}\setlength{\parsep}{0in}\setlength{\itemsep}{0in}\setlength{\parskip}{0in}\usecounter\@enumctr\def\makelabel##1{\hss\llap{##1}}}%
\fi}
\let\endtightenumerate =\endlist

\titlespacing{\paragraph}{0pt}{*1}{*3}

%Control whether comments render or not in the generated .pdf
\newboolean{draft}
\setboolean{draft}{true}
\ifthenelse{\boolean{draft}}
{
  \newcommand{\pgb}[1]{\textcolor{orange}{[PGB: #1]}}
  \newcommand{\as}[1]{\textcolor{green}{[AS: #1]}}
  \newcommand{\msf}[1]{\textcolor{blue}{[MSF: #1]}}
  \newcommand{\mgfd}[1]{\textcolor{purple}{[MGFD: #1]}}
  \newcommand{\editmark}{\marginpar{\colorbox{yellow}{\sf Update}}}
  \newcommand{\reminder}[1]{\textcolor{red}{\textit{\textsc{#1}}}}
}
{
  \newcommand{\pgb}[1]{}
  \newcommand{\as}[1]{}
  \newcommand{\msg}[1]{}
  \newcommand{\mgfd}[1]{}
  \newcommand{\editmark}{}
  \newcommand{\reminder}[1]{}
}

\rhead{}
\chead{}
\lhead{}
\rhead{NSF Proposal Header Title}
%\renewcommand{\headrulewidth}{0pt}
%\renewcommand{\footrulewidth}{0pt}

\newcommand{\nhalf}{$N_\frac{1}{2}$}

\newcommand{\smartincludepdf}[3]{
      % Set headers and TOC entry
      \markright{#1}

      \IfFileExists{#2.pdf}{
         % Use fancy page style to put headers and page numbers on the
         % included pages for easier navigation.  With luck, the included
         % pages will not overlap these marks.
        \addcontentsline{toc}{section}{\numberline {}#1}
         \includepdf[pages=-,pagecommand={\thispagestyle{#3}}]{#2}
      }{
         % If the desired PDF file doesn't exist, warn about it but continue
        \addcontentsline{toc}{section}{\numberline {}#1 {\color{red}*** MISSING! ***}}
         \begin{center}
            \textbf{\color{red}\Large No file found for '#1'!\\
                    Tried #2.tex and #2.pdf} \\
         \end{center}
      }
}

\hyphenation{sto-chas-tic} \pagestyle{plain}

\def\I{{I}}
\def\Tc{{T_{compute,}}}
\def\Ti{{T_{interference,}}}
\def\Te{{T_{execution,}}}
\def\p{\epsilon}
\def\pp{\phi}
\def\fc{{f_{c}}}
\def\ff{{f_{int}}}
\def\fe{{f_{e}}}
\def\FX{{F_{X_P}}}
\def\Fe{{F_e}}
\def\X{{X}}
\def\J{{J}}
\def\FXb{{F_{\Te|\Ti_1}}}
\def\Tes{{T_{execution}}}

\newcommand{\spp}{\vspace{1.5ex}}
\newcommand{\be}{\begin{equation}}
\newcommand{\ee}{\end{equation}}
\newcommand{\bdm}{\begin{displaymath}}
\newcommand{\edm}{\end{displaymath}}

\newcommand{\bea}{\begin{eqnarray}}
\newcommand{\eea}{\end{eqnarray}}
\newcommand{\beas}{\begin{eqnarray*}}
\newcommand{\eeas}{\end{eqnarray*}}

\newcommand{\sbea}{\nopagebreak[3]\samepage\begin{eqnarray}}
\newcommand{\seea}{\end{eqnarray}\pagebreak[0]}
\newcommand{\sbeas}{\nopagebreak[3]\samepage\begin{eqnarray*}}
\newcommand{\seeas}{\end{eqnarray*}\pagebreak[0]}

\def\intii{\int_{-\infty}^\infty}
\def\intzi{\int_0^\infty}
\def\dint{\mathop{\int\!\!\int}\limits}

\def\lb{\label} % WARNING don't use \let here;
%                 it interferes with dlabels.sty
\newcommand{\er}[1]{{\rm(\ref{#1})}}
\newcommand{\sler}[1]{{\sl(\ref{#1})}}
\newcommand{\nr}[1]{{\rm\ref{#1}}}
\newcommand{\slnr}[1]{{\sl\ref{#1}}}
\newcommand{\rmbox}[1]{\mbox{\rm #1}}

\let\la=\langle
\let\ra=\rangle

\def\ebox{\raisebox{-.24ex}{\mbox{\large$\Box$}}}
\def\QED{\hfill\ebox}
\def\eQED{\gdef\@eqnnum{\ebox}}
\def\fix{\gdef\@eqnnum{{{\rm (\theequation)}}}\addtocounter{equation}{-1}}

\def\mycircs{\raisebox{.123ex}{\hspace*{.05em}$\scriptstyle\bigcirc$}}
\def\nstwepts{{\mbox{\scriptsize$\not\mathrel{\kern-.9pt\mycircs}$}}}

\def\mycircd{\raisebox{.08ex}{\kern-.02em\footnotesize$\bigcirc$}}
\def\nstweptd{{\mbox{$\not\mathrel{\mycircd}$}}}

\def\nstwept{{\mathchoice\nstweptd\nstweptd\nstwepts\nstwepts}}

%\def\nstenpt{{\mbox{$\not\mathrel{\raisebox{.2pt}{\kern-.5pt\mycircd}}$}}}
\let\nstenpt=\nstwept

\let\ns=\nstwept

% tenpt footnotesize \ns 
\def\nsft{\mbox{\scriptsize$\not\mathrel{\kern-.9pt\mycircs}$}}

\def\bgz{\makebox[0pt][c]{\huge 0}}

\def\conv{\mathop{\raisebox{-.15ex}{$\ast$}\kern-.62em{\scriptstyle\bigcirc}}}

\def\indep{\mathop{\bot\kern-.6em\bot}}

\def\DIC{{\hbox{\rm\kern.2em\raise.36ex%
\hbox{$\scriptstyle |$}\kern-.4em C}}}
\def\SIC{{\hbox{\scriptsize\rm\kern.2em\raise.4ex%
\hbox{$\scriptscriptstyle |$}\kern-.4em C}}}
\def\IC{{\mathchoice\DIC\DIC\SIC\SIC}}

\def\DIQ{{\hbox{\rm\kern.2em\raise.4ex%
\hbox{$\scriptstyle |$}\kern-.4em Q}}}
\def\SIQ{{\hbox{\scriptsize\rm\kern.2em\raise.4ex%
\hbox{$\scriptscriptstyle |$}\kern-.4em Q}}}
\def\IQ{{\mathchoice\DIQ\DIQ\SIQ\SIQ}}

\def\IR{{\rm I\!R}}
\def\IP{{\rm I\!P}}
\def\IE{{\rm I\!E}}
\def\IN{{\rm I\!N}}
\def\Ind{{\bf 1}}

\def\DZZ{{\hbox{\sf Z\kern-.41em Z}}}
\def\SZZ{{\hbox{\scriptsize\sf Z\kern-.41em Z}}}
\def\ZZ{{\mathchoice\DZZ\DZZ\SZZ\SZZ}}
\let\IZ=\ZZ

\def\bE{{\sf E}}
\def\bEf{\hbox{\footnotesize\sf E}}
\def\bP{{\sf P}}
%\def\sP{\hbox{\kern-.15em\raise.5ex\hbox{\large$\wp$}}}
\def\sP{\hbox{\kern-.05em\raise.5ex\hbox{\large$\wp$}}}
\def\sPf%
{\hbox{\kern-.05em\raise.5ex\hbox{\normalsize$\wp$}}} % \footnotesize \sP
\def\Pr{{\rm Pr}}
\def\sB{{\cal B}}
\def\sF{{\cal F}}
\def\sG{{\cal G}}

%\def\de{  \buildrel {\rm \bigtriangleup} \over = }
\def\de{\buildrel{\scriptscriptstyle\bigtriangleup}\over=}
\def\cde{\mathrel{:=}}
\def\dec{\mathrel{=:}}
\let\oliminf=\liminf
\let\olimsup=\limsup
\def\lim@sup{\mathop{\overline{\rm lim}}}
\def\lim@inf{\mathop{\underline{\rm lim}}}
\let\limsup=\lim@sup
\let\liminf=\lim@inf
\def\argmin{\mathop{\rm argmin}}
\def\argmax{\mathop{\rm argmax}}
\def\mspan{\mathop{\rm span}\nolimits}
\def\tr{\mathop{\rm tr}\nolimits}
\def\re{\mathop{\rm Re}\nolimits}
\def\im{\mathop{\rm Im}\nolimits}
\def\sgn{\mathop{\rm sgn}\nolimits}
\def\co{\mathop{\rm co}\nolimits}
\def\cl{\mathop{\rm cl}\nolimits}
\def\clco{\overline{\co}}
\let\downto=\downarrow

\def\Dbt{{\raisebox{-.8ex}{\mbox{\LARGE\sf X}}}}
\def\Dpbt{{\raisebox{-.4ex}{\mbox{\large\sf X}}}}
\def\Sbt{{\raisebox{-.4ex}{\mbox{\sf X}}}}
\def\bigtimes{\mathop{\mathchoice\Dbt\Dpbt\Sbt\Sbt}}

\def\ds{\displaystyle}
\def\ts{\textstyle}
%\def\lrt{\mathop{\raisebox{-.5ex}{$\buildrel{\ds >}\over<$}}}
%\def\LRT{\ds {{{H_1} \atop {\ds >}} \atop {{\ds <} \atop {H_0}}}}
\def\LRT#1#2{\ds {{{#1} \atop {\ds >}} \atop {{\ds <} \atop {#2}}}}

\def\mx{{\rm max}}
\def\mn{{\rm min}}
\def\crs{\times}
\def\half{{\textstyle{ 1 \over 2}}}
\def\ninv{(1/n)}

\def\same{\rule{2.3em}{0.3pt}\,, }

\def\f@left#1{\makebox[0pt][l]{$\displaystyle#1$}}
% \def\f@oleft[#1]#2{\makebox[0pt][l]{\hspace*{-#1}$\displaystyle#2$}}
\def\f@oleft[#1]#2{\makebox[0pt][l]{$\displaystyle#2$}\hspace*{#1}}
\def\lefteqn{\@ifnextchar[{\f@oleft}{\f@left}}

\def\eqbox#1#2{\makebox[#1]{$\displaystyle#2$}}

\let\refsize=\small
\def\call#1{}
\def\singlespace{}

\outer\def\fl{\@ifnextchar[{\@opfl}{\@fl}}
\def\@opfl[#1]#2{\fl@{#1}{#2}}
\def\@fl#1{\fl@{-2.5ex}{#1}}
\def\fl@#1{}
\def\UC{}

%--------------------------------------------------------------------
%
% Define my version of \mathpalette.  Example:  \def\bx{\pal{\bf x}}
%
\def\pal#1{{\mathchoice{#1}{#1}{\hbox{\scriptsize#1}}{\hbox{\tiny#1}}}}
%
% If font does not exist in \tiny size, use \pals instead.
% Example: \def\sX{\pals{\sf X}}
%
\def\pals#1{{\mathchoice{#1}{#1}{\hbox{\scriptsize#1}}{\hbox{\scriptsize#1}
}}}
%--------------------------------------------------------------------
\iftrue
\def\optlist{}

\def\list#1#2{\ifnum \@listdepth >5\relax \@toodeep
     \else \global\advance\@listdepth\@ne \fi
  \rightmargin \z@ \listparindent\z@ \itemindent\z@
  \csname @list\romannumeral\the\@listdepth\endcsname
  \def\@itemlabel{#1}\let\makelabel\@mklab \@nmbrlistfalse #2\optlist\relax
  \@trivlist
  \parskip\parsep \parindent\listparindent
  \advance\linewidth -\rightmargin \advance\linewidth -\leftmargin
  \advance\@totalleftmargin \leftmargin
  \parshape \@ne \@totalleftmargin \linewidth
  \ignorespaces}


\def\mylabel#1{\def\@itemlabel{#1}}
\fi
%--------------------------------------------------------------------


\author{}
\date{}

\begin{document}

\cfoot{Desc-\thepage}
\pagestyle{fancy}
\addtolength{\baselineskip}{-1pt}
\newpage
\setcounter{page}{1}

% Full Proposal

\noindent{\Large{\bf Project Title:} CRI: II-EN: Research Infrastructure for Continuous Analysis of High-Intensity Data Streams }

\noindent{\Large{\bf Project Type:} II-EN}

\noindent{\Large{\bf APPROVED PRELIMINARY PROPOSAL CURRENTLY UPLOADED AS PLACEHOLDER FOR DESCRIPTION IN DEVELOPMENT}}

\section*{Project Personnel}
\subsection*{Lead Institution:}

\begin{minipage}{6in}
\begin{multicols}{2}
\begin{itemize}
\item Prof.~Patrick G. Bridges; PI
\item Prof.~Jedidiah Crandall; Co-PI
\item Prof.~Trilce Estrada; Co-PI
\item Prof.~Abdulluh Mueen; Co-PI
\item Prof. Marios Pattichis; Co-PI
\item Hussein Al-Azzawi; Senior Personnel
\end{itemize}
\end{multicols}
\end{minipage}

\vspace*{0.1in}
\paragraph{Institutions submitting collaborative proposals:} No Collaborative Institutions

\paragraph{Other Collaborators:} No External Collaborators

\vspace*{0.2in}
\noindent{\Large{\bf CISE core division:} CNS, CCF}

\vspace*{0.1in}
\noindent{\Large{\bf Projected budget total:} \$616,956}

\vspace*{0.1in}
\noindent{\Large{\bf Keywords:} Streaming data analysis, computer systems and networks, machine learning}

\vspace*{0.1in}
\section{Infrastructure to be developed/deployed}

We propose to acquire enhance existing cyber-infrastructure to support novel computer science research 
into online analysis of large-scale heterogeneous data and systems, particularly continuous
analysis of high-intensity data streams. The proposed system will augment existing storage and networking
infrastructure for data analysis offline with diverse networking, computational, and storage capabilities; 
storage diversity is particularly important given the wide range of research the system will support. 
The system will include a combination of many-core computing nodes, high-bandwidth parallel storage, 
low-latency non-volatile storage, and high-reliability enterprise storage to support, respectively,
high-speed analysis, buffering large amounts in-flight data, caching reference data sets, and preserving 
key results and supporting data. 

We specifically propose to enhance existing near-line storage/high-speed networking
infrastructure with:
\begin{itemize}
\item Additional high-speed networking connectivity to facilitate handling additional data streams;
\item 500TB of high-speed InfiniBand-connected parallel file and object storage;
\item 50TB of all-flash or other non-volatile, low-latency storage;
\item 100TB of enterprise-grade storage;
\item 4 many-core cluster computing nodes (for example Intel Xeon Phi Knight's Mill or NVIDIA P100 GPU systems); and
\item 5 years of UNM computer services support for the acquired systems.
\end{itemize}

\section{CISE Research Focus}

Supported CISE research will focus on the continuous analysis of high-intensity 
streaming data sources, including locally and regionally generated data sets. 
Research in handling and extracting research insights from the deluge of available data
is pervasive across CISE in areas such as computer system 
design and optimization, network analysis, social media analysis, and 
machine learning. The project PIs perform CISE-related research in each of these areas,
that require high-speed computational analysis of large amounts of streaming data.

Prof.~Patrick G. Bridges will lead this project both in his capacity as Director of the
UNM Center for Advanced Research Computing (CARC), which will house and support the
infrastructure, and as a CISE researcher. In the first role, Prof.~Bridges will 
oversee the specification, acquisition, installation, and support of the resulting 
infrastructure in collaboration with other project PIs. The project will support Prof.~Bridges's 
research on the optimization of HPC and Big Data applications and system software,
research which requires the analysis of streaming performance data from 
large-scale computing systems.

Co-PIs Crandall, Estrada, Mueen, and Pattichis will leverage the system to support their 
respective research in analysis of streaming data. PI Crandall's research in this area
focuses on using side-channel network analysis techniques to discover and map hidden
Internet features. His current research uses the existing infrastructure for the offline analysis
of network traces to infer network characteristics; the proposed enhancement will enable
continuous online analysis of these characteristics. PI Estrada's and Pattichis's research 
focuses on machine learning abstractions for the modeling and analysis of scientific and 
imaging data; a 100+TB medical forensics image database currently being assembled at CARC 
is one data-set of interest for this research. PI Mueen's research focuses on the long-term 
analysis of multiple complex social media data streams, including novel bot detection 
techniques in Twitter data streams.

Ongoing technical support for the infrastructure will be provided by CARC staff 
member Hussein Al-Azzawi under the supervision of Prof.~Bridges. Mr. Al-Azzawi 
currently supports the existing storage and networking infrastructure that will
be enhanced.

\section{Sample Research Project}
Prof.~Patrick Bridges's research will leverage the acquired infrastructure to 
research the optimization of resource allocation in large-scale and strategic computing systems 
using high-fidelity stochastic application performance models. These models of HPC and 
Big Data applications are developed by analyzing streams of communication and processor 
performance samples from running computational systems; the resulting models are used 
to optimize and adapt resource allocation in running systems.

\section{Relevance to CISE}

Each of the project PIs (Bridges, Crandall, Estrada, Mueen, and Pattichis) are established
CISE researchers who will, as described above, provide continual feedback on the development
and deployment of the infrastructure. The resulting infrastructure will benefit CISE research
communities in computer systems research, computer networking, parallel and distributed computing, 
machine learning, image analysis, and intelligent systems. The existing storage/networking system that 
will provide the foundation for this award was acquired in 2015 under CNS award 1518878.
 
% Bibliography
%\newpage
%\cfoot{Ref-\thepage}
%\setcounter{page}{1}
%\bibliographystyle{plain}
%\bibliography{refs.bib}

\end{document}
